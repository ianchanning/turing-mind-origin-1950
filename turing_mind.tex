\documentclass[10pt]{article} % Starting with 10pt as a base

% --- Preamble: Packages and Global Settings ---
\usepackage[utf8]{inputenc} % Input encoding (important for direct copy-paste of text)
\usepackage[T1]{fontenc}    % Font encoding for proper glyphs and hyphenation
\usepackage{tgschola}       % For TeX Gyre Schola font
\usepackage{amsmath}        % For math environments (e.g., align)
\usepackage{amssymb}        % For additional math symbols
\usepackage{graphicx}       % For including images (if any)
\usepackage{setspace}       % For line spacing control
\usepackage{titlesec}       % For fine-grained control over section titles
\usepackage{fancyhdr}       % For custom headers and footers
\usepackage{soul}           % For letter spacing (\so)
\usepackage{enumitem}       % For customizing list environments (e.g., (i), (ii))
\usepackage{textcomp}       % For \textquotesingle

% --- Page Geometry (from your precise measurements) ---
\usepackage[
    paperwidth=141mm,   % Width of the physical paper
    paperheight=231mm,  % Height of the physical paper
    left=22mm,          % Distance from left paper edge to text block
    right=22mm,         % Distance from right paper edge to text block
    top=22mm,           % Distance from top paper edge to text block
    bottom=34mm,        % Distance from bottom paper edge to text block
    % With these, textwidth=97mm and textheight=175mm are automatically calculated by geometry.
]{geometry}

% --- Header and Footer Configuration ---
% Define default page style using fancyhdr
\pagestyle{fancy}
\fancyhf{} % Clear all header and footer fields

% First page style (unique header/footer)
\fancypagestyle{firstpage}{
    \fancyhf{} % Clear existing settings for this page
    % Confirmed from scan: VOL. LIX. is on one line, No. 236.] is on the next, slightly indented.
    % The font size is indeed very small.
    \fancyhead[L]{\normalfont\fontsize{6.5pt}{7.5pt}\selectfont VOL. LIX.\\ \hspace*{0.5em}No. 236.]} % Top left (Vol/No info), with indent on second line
    \fancyhead[R]{\normalfont\fontsize{6.5pt}{7.5pt}\selectfont [October, 1950} % Top right (Date)
    \fancyfoot[L]{\normalfont\small 28} % Bottom left (Journal series number)
    \fancyfoot[R]{\normalfont\small 433} % Bottom right (Page number)
    \renewcommand{\headrulewidth}{0pt} % No rule under header
    \renewcommand{\footrulewidth}{0pt} % No rule above footer
}

% General page style for subsequent pages (alternating running heads and page numbers)
\fancypagestyle{mainpage}{
    \fancyhf{} % Clear existing settings
    % Left Even pages (e.g., page 434): Page number on left, Author on right
    \fancyhead[LE]{\normalfont\small \thepage \ A. M. TURING :} % Left Even
    % Right Odd pages (e.g., page 435): Article Title on left, Page number on right
    \fancyhead[RO]{\normalfont\small COMPUTING MACHINERY AND INTELLIGENCE \thepage} % Right Odd
    \renewcommand{\headrulewidth}{0pt} % No rule under header
    \renewcommand{\footrulewidth}{0pt} % No rule above footer
}
\pagestyle{mainpage} % Set this as the default style for all pages (first page overrides it)

% --- Customizing Article Title and Section Titles ---
% Custom command for the main article title (I. COMPUTING MACHINERY AND INTELLIGENCE)
% Now centered and slightly larger
\newcommand{\articletitle}[1]{%
    \vspace*{0.7cm} % Adjust vertical spacing above title
    \begin{center}% Center the title
        {\normalfont\LARGE\textbf{#1}}% Slightly larger, bold, uppercase
    \end{center}%
    \par\vspace{0.2cm}%
}

% Custom command for the author line (BY A. M. TURING)
% Now centered, and explicitly in small caps
\newcommand{\articleauthor}[1]{%
    \begin{center}% Center the author
        {\normalfont#1}% Content passed will be small caps
    \end{center}%
    \par\vspace{0.5cm}%
}

% Custom command for Q&A dialogue turns (flush left, non-bold speaker)
\newcommand{\dialogueturn}[2]{%
    \noindent\normalfont #1: #2\par % Speaker normal weight, text normal, flush left
    \vspace{0.2\baselineskip}% Small vertical space after each turn
}

% Global line spacing for body text
\setstretch{1.15} % This value might need fine-tuning for exact line count per page.

% --- Document Start ---
\begin{document}
\thispagestyle{firstpage} % Apply the unique first page style to the first page only

% --- Top Journal Information (handled by fancypagestyle{firstpage}) ---

% --- Main Journal Title Block ---
\vspace*{1.2cm} % Adjust this to push "MIND" down to match the original
\begin{center}
    % Applying letter spacing with \so from 'soul' package
    % Keeping it bold and huge, as that matches the original's heavy stroke.
    {\Huge\textbf{\so{MIND}}}
\end{center}

\vspace*{0.5cm} % Space between MIND and A QUARTERLY REVIEW
\begin{center}
    {\large A QUARTERLY REVIEW} \\
    \vspace{0.1cm} % Small space between lines
    {\normalsize OF} \\
    \vspace{0.1cm} % Small space between lines
    {\large PSYCHOLOGY AND PHILOSOPHY}
\end{center}

% Decorative separator
\vspace*{0.5cm} % Space below philosophy text
\begin{center}
    \rule{3cm}{0.75pt} % Horizontal rule mimicking the decorative element
\end{center}
\vspace*{0.7cm} % Space below separator

% --- Article Title and Author ---
\articletitle{I.\textemdash COMPUTING MACHINERY AND INTELLIGENCE} % Use \textemdash for the long dash (em-dash)
\articleauthor{\textsc{BY A. M. TURING}} % Now BY A. M. TURING is passed as small caps

% --- Article Content Starts ---
% Section 1
% Section header: number normal, rest italic.
\noindent\normalfont 1. \textit{The Imitation Game.}
\vspace{0.5\baselineskip} % Small space below the section title

I \textsc{PROPOSE} to consider the question, `Can machines think?' This should begin with definitions of the meaning of the terms `machine' and `think'. The definitions might be framed so as to reflect so far as possible the normal use of the words, but this attitude is dangerous. If the meaning of the words `machine' and `think' are to be found by examining how they are commonly used it is difficult to escape the conclusion that the meaning and the answer to the question, `Can machines think?' is to be sought in a statistical survey such as a Gallup poll. But this is absurd. Instead of attempting such a definition I shall replace the question by another, which is closely related to it and is expressed in relatively unambiguous words.

The new form of the problem can be described in terms of a game which we call the `imitation game'. It is played with three people, a man (A), a woman (B), and an interrogator (C) who may be of either sex. The interrogator stays in a room apart from the other two. The object of the game for the interrogator is to determine which of the other two is the man and which is the woman. He knows them by labels X and Y, and at the end of the game he says either `X is A and Y is B' or `X is B and Y is A'. The interrogator is allowed to put questions to A and B thus:

\vspace{0.5\baselineskip} % Space before dialogue starts
\dialogueturn{C}{Will X please tell me the length of his or her hair ?}% Changed to new command
Now suppose X is actually A, then A must answer. It is A's
% --- End of Original PDF page 433 (my page 1) ---

% --- Content for Original PDF page 434 (my page 2) ---
object in the game to try and cause C to make the wrong identification. His answer might therefore be
\vspace{0.2\baselineskip} % Small space before quoted answer
``My hair is shingled, and the longest strands are about nine inches long.''\par % Corrected curly quotes
\vspace{0.5\baselineskip} % Space after quoted answer

In order that tones of voice may not help the interrogator the answers should be written, or better still, typewritten. The ideal arrangement is to have a teleprinter communicating between the two rooms. Alternatively the question and answers can be repeated by an intermediary. The object of the game for the third player (B) is to help the interrogator. The best strategy for her is probably to give truthful answers. She can add such things as ``I am the woman, don't listen to him!'' to her answers, but it will avail nothing as the man can make similar remarks.

We now ask the question, `What will happen when a machine takes the part of A in this game?'` Will the interrogator decide wrongly as often when the game is played like this as he does when the game is played between a man and a woman? These questions replace our original, `Can machines think?'`

% Section 2
% Section header: number normal, rest italic.
\vspace{0.5\baselineskip} % Space before new section
\noindent\normalfont 2. \textit{Critique of the New Problem.}
\vspace{0.5\baselineskip} % Small space below the section title

As well as asking, `What is the answer to this new form of the question,` one may ask, `Is this new question a worthy one to investigate?'` This latter question we investigate without further ado, thereby cutting short an infinite regress.

The new problem has the advantage of drawing a fairly sharp line between the physical and the intellectual capacities of a man. No engineer or chemist claims to be able to produce a material which is indistinguishable from the human skin. It is possible that at some time this might be done, but even supposing this invention available we should feel there was little point in trying to make a `thinking machine'` more human by dressing it up in such artificial flesh. The form in which we have set the problem reflects this fact in the condition which prevents the interrogator from seeing or touching the other competitors, or hearing their voices. Some other advantages of the proposed criterion may be shown up by specimen questions and answers. Thus:

\vspace{0.5\baselineskip} % Space before dialogue
\dialogueturn{Q}{Please write me a sonnet on the subject of the Forth Bridge.}% Changed to new command
\dialogueturn{A}{Count me out on this one. I never could write poetry.}% Changed to new command
\dialogueturn{Q}{Add 34957 to 70764}% Changed to new command
\dialogueturn{A}{(Pause about 30 seconds and then give as answer) 105621.}% Changed to new command
\dialogueturn{Q}{Do you play chess ?}% Changed to new command
\dialogueturn{A}{Yes.}% Changed to new command
% --- End of Original PDF page 434 (my page 2) ---

% --- Content for Original PDF page 435 (my page 3) ---
% The original page 435 starts with the rest of the chess dialogue.
\vspace{0.5\baselineskip} % Add a bit of space before the continuation of Q&A
\dialogueturn{Q}{I have K at my K1, and no other pieces. You have only K at K6 and R at R1. It is your move. What do you play ?}% Changed to new command
\dialogueturn{A}{(After a pause of 15 seconds) R-R8 mate.}% Changed to new command

% Section 3 now correctly placed after the full Q&A from page 435
\vspace{0.5\baselineskip} % Space before new section
\noindent\normalfont 3. \textit{The Machines concerned in the Game.}
\vspace{0.5\baselineskip} % Small space below the section title

The question which we put in \S 1 will not be quite definite until we have specified what we mean by the word `machine'. It is natural that we should wish to permit every kind of engineering technique to be used in our machines. We also wish to allow the possibility than an engineer or team of engineers may construct a machine which works, but whose manner of operation cannot be satisfactorily described by its constructors because they have applied a method which is largely experimental. Finally, we wish to exclude from the machines men born in the usual manner. It is difficult to frame the definitions so as to satisfy these three conditions. One might for instance insist that the team of engineers should be all of one sex, but this would not really be satisfactory, for it is probably possible to rear a complete individual from a single cell of the skin (say) of a man. To do so would be a feat of biological technique deserving of the very highest praise, but we would not be inclined to regard it`as a case of `constructing a thinking machine'. This prompts us to abandon the requirement that every kind of technique should be permitted. We are the more ready to do so in view of the fact that the present interest in `thinking machines' has been aroused by a particular kind of machine, usually called an `electronic computer' or `digital computer'. Following this suggestion we only permit digital computers to take part in our game.

This restriction appears at first sight to be a very drastic one. I shall attempt to show that it is not so in reality. To do this necessitates a short account of the nature and properties of these computers.

It may also be said that this identification of machines with digital computers, like our criterion for `thinking', will only be unsatisfactory if (contrary to my belief), it turns out that digital computers are unable to give a good showing in the game.

There are already a number of digital computers in working order, and it may be asked, `Why not try the experiment straight away? It would be easy to satisfy the conditions of the game. A number of interrogators could be used, and statistics compiled to show how often the right identification was given.' The short answer is that we are not asking whether all digital computers would do well in the game nor whether the computers at present available would do well, but whether there are imaginable computers which would do well. But this is only the short answer. We shall see this question in a different light later.
% --- End of Original PDF page 436 (my page 4) ---

% --- Content for Original PDF page 437 (my page 5) ---
% Section 4
\vspace{0.5\baselineskip} % Space before new section
\noindent\normalfont 4. \textit{Digital Computers.}
\vspace{0.5\baselineskip} % Small space below the section title

The idea behind digital computers may be explained by saying that these machines are intended to carry out any operations which could be done by a human computer. The human computer is supposed to be following fixed rules; he has no authority to deviate from them in any detail. We may suppose that these rules are supplied in a book, which is altered whenever he is put on to a new job. He has also an unlimited supply of paper on which he does his calculations. He may also do his multiplications and additions on a `desk machine', but this is not important.
If we use the above explanation as a definition we shall be in danger of circularity of argument. We avoid this by giving an outline of the means by which the desired effect is achieved. A digital computer can usually be regarded as consisting of three parts:
\begin{enumerate}[label=(\roman*)]
    \item Store.
    \item Executive unit.
    \item Control.
\end{enumerate}
The store is a store of information, and corresponds to the human computer's paper, whether this is the paper on which he does his calculations or that on which his book of rules is printed. In so far as the human computer does calculations in his head a part of the store will correspond to his memory.
The executive unit is the part which carries out the various individual operations involved in a calculation. What these individual operations are will vary from machine to machine. Usually fairly lengthy operations can be done such as `Multiply 3540675445 by 7076345687'` but in some machines only very simple ones such as `Write down 0'` are possible.
We have mentioned that the `book of rules'` supplied to the computer is replaced in the machine by a part of the store. It is then called the `table of instructions'`. It is the duty of the control to see that these instructions are obeyed correctly and in the right order. The control is so constructed that this necessarily happens.
The information in the store is usually broken up into packets of moderately small size. In one machine, for instance, a packet might consist of ten decimal digits. Numbers are assigned to the parts of the store in which the various packets of information are stored, in some systematic manner. A typical instruction might say-
\vspace{0.2\baselineskip} % Small space before quoted instruction
``Add the number stored in position 6809 to that in 4302 and put the result back into the latter storage position.''\par % Corrected curly quotes
\vspace{0.5\baselineskip} % Space after quoted instruction

Needless to say it would not occur in the machine expressed in English. It would more likely be coded in a form such as 6809430217. Here 17 says which of various possible operations is to be performed on the two numbers. In this case the operation is that described above, viz. ``Add the number....''` It will be noticed that the instruction takes up 10 digits and so forms one packet of information, very conveniently. The control will normally take the instructions to be obeyed in the order of the positions in which they are stored, but occasionally an instruction such as
% --- End of Original PDF page 437 (my page 5) ---

% --- Content for Original PDF page 438 (my page 6) ---
\vspace{0.5\baselineskip} % Space before quoted instruction
``Now obey the instruction stored in position 5606, and continue from there.''\par % Corrected curly quotes
\vspace{0.5\baselineskip} % Space after quoted instruction
may be encountered, or again
\vspace{0.5\baselineskip} % Space before quoted instruction
``If position 4505 contains 0 obey next the instruction stored in 6707, otherwise continue straight on.''\par % Corrected curly quotes
\vspace{0.5\baselineskip} % Space after quoted instruction
Instructions of these latter types are very important because they make it possible for a sequence of operations to be repeated over and over again until some condition is fulfilled, but in doing so to obey, not fresh instructions on each repetition, but the same ones over and over again. To take a domestic analogy. Suppose Mother wants Tommy to call at the cobbler's every morning on his way to school to see if her shoes are done, she can ask him afresh every morning. Alternatively she can stick up a notice once and for all in the hall which he will see when he leaves for school and which tells him to call for the shoes, and also to destroy the notice when he comes back if he has the shoes with him.

The reader must accept it as a fact that digital computers can be constructed, and indeed have been constructed, according to the principles we have described, and that they can in fact mimic the actions of a human computer very closely.

The book of rules which we have described our human computer as using is of course a convenient fiction. Actual human computers really remember what they have got to do. If one wants to make a machine mimic the behaviour of the human computer in some complex operation one has to ask him how it is done, and then translate the answer into the form of an instruction table. Constructing instruction tables is usually described as `programming'. To `programme a machine to carry out the operation A'` means to put the appropriate instruction table into the machine so that it will do A.

An interesting variant on the idea of a digital computer is a `digital computer with a random element'`. These have instructions involving the throwing of a die or some equivalent electronic process; one such instruction might for instance be, ``Throw the die and put the resulting number into store 1000''.` Sometimes such a machine is described as having free will (though I would not use this phrase myself). It is not normally possible to determine from observing a machine whether it has a random element, for a similar effect can be produced by such devices as making the choices depend on the digits of the decimal for $\pi$.

Most actual digital computers have only a finite store. There is no theoretical difficulty in the idea of a computer with an unlimited store. Of course only a finite part can have been used at any one time. Likewise only a finite amount can have been
% --- End of Original PDF page 438 (my page 6) ---

% --- Content for Original PDF page 439 (my page 7) ---
constructed, but we can imagine more and more being added as required. Such computers have special theoretical interest and will be called infinitive capacity computers.

The idea of a digital computer is an old one. Charles Babbage, Lucasian Professor of Mathematics at Cambridge from 1828 to 1839, planned such a machine, called the Analytical Engine, but it was never completed. Although Babbage had all the essential ideas, his machine was not at that time such a very attractive prospect. The speed which would have been available would be definitely faster than a human computer but something like 100 times slower than the Manchester machine, itself one of the slower of the modern machines. The storage was to be purely mechanical, using wheels and cards.

The fact that Babbage's Analytical Engine was to be entirely mechanical will help us to rid ourselves of a superstition. Importance is often attached to the fact that modern digital computers are electrical, and that the nervous system also is electrical. Since Babbage's machine was not electrical, and since all digital computers are in a sense equivalent, we see that this use of electricity cannot be of theoretical importance. Of course electricity usually comes in where fast signalling is concerned, so that it is not surprising that we find it in both these connections. In the nervous system chemical phenomena are at least as important as electrical. In certain computers the storage system is mainly acoustic. The feature of using electricity is thus seen to be only a very superficial similarity. If we wish to find such similarities we should look rather for mathematical analogies of function.

% Section 5
\vspace{0.5\baselineskip} % Space before new section
\noindent\normalfont 5. \textit{Universality of Digital Computers.}
\vspace{0.5\baselineskip} % Small space below the section title

The digital computers considered in the last section may be classified amongst the `discrete state machines'. These are the machines which move by sudden jumps or clicks from one quite definite state to another. These states are sufficiently different for the possibility of confusion between them to be ignored. Strictly speaking there are no such machines. Everything really moves continuously. But there are many kinds of machine which can profitably be thought of as being discrete state machines. For instance in considering the switches for a lighting system it is a convenient fiction that each switch must be definitely on or definitely off. There must be intermediate positions, but for most purposes we can forget about them. As an example of a discrete state machine we might consider a wheel which clicks round through 120$^\circ$ once a second, but may be stopped by a lever which can be operated from outside; in addition a lamp is to light in one of the positions of the wheel. This machine could be described abstractly as follows. The internal state of the machine (which is described by the position of the wheel) may be q$_1$, q$_2$ or q$_3$. There is an input signal i$_0$, or i$_1$ (position of lever). The internal state at any moment is determined by the last state and input signal according to the table
\vspace{0.5\baselineskip} % Space before table

\begin{center}
    \normalfont % Ensure normal font within table
    \textbf{Last State} \\
    \begin{tabular}{|c|c|c|c|}
        \hline
        \multicolumn{1}{|c|}{} & \textbf{q$_1$} & \textbf{q$_2$} & \textbf{q$_3$} \\
        \hline
        \textbf{i$_0$} & q$_2$ & q$_3$ & q$_1$ \\
        \cline{2-4}
        \textbf{Input} & & & \\ % This row is for "Input" label, so it's empty otherwise
        \cline{1-1} % Only draw line under "Input"
        \textbf{i$_1$} & q$_1$ & q$_2$ & q$_3$ \\
        \hline
    \end{tabular}
\end{center}
\vspace{0.5\baselineskip} % Space after table

The output signals, the only externally visible indication of the internal state (the light) are described by the table
\vspace{0.5\baselineskip} % Space before table

\begin{center}
    \normalfont % Ensure normal font within table
    \begin{tabular}{ccc}
        \textbf{State} & \textbf{q$_1$} & \textbf{q$_2$} \textbf{q$_3$} \\
        \textbf{Output} & \textbf{o$_0$} & \textbf{o$_0$} \textbf{o$_1$} \\
    \end{tabular}
\end{center}
\vspace{0.5\baselineskip} % Space after table

This example is typical of discrete state machines. They can be described by such tables provided they have only a finite number of possible states.

It will seem that given the initial state of the machine and the input signals it is always possible to predict all future states. This is reminiscent of Laplace's view that from the complete state of the universe at one moment of time, as described by the positions and velocities of all particles, it should be possible to predict all future states. The prediction which we are considering is, however, rather nearer to practicability than that considered by Laplace. The system of the `universe as a whole'' is such that quite small errors in the initial conditions can have an overwhelming effect at a later time. The displacement of a single electron by a billionth of a centimetre at one moment might make the difference between a man being killed by an avalanche a year later, or escaping. It is an essential property of the mechanical systems which we have called `discrete state machines'' that this phenomenon does not occur. Even when we consider the actual physical machines instead of the idealised machines, reasonably accurate knowledge of the state at one moment yields reasonably accurate knowledge any number of steps later.
% --- End of Original PDF page 440 (my page 8) ---

% --- Content for Original PDF page 441 (my page 9) ---
As we have mentioned, digital computers fall within the class of discrete state machines. But the number of states of which such a machine is capable is usually enormously large. For instance, the number for the machine now working at Manchester is about $2^{165,000}$, i.e.~about $10^{50,000}$. Compare this with our example of the clicking wheel described above, which had three states. It is not difficult to see why the number of states should be so immense. The computer includes a store corresponding to the paper used by a human computer. It must be possible to write into the store any one of the combinations of symbols which might have been written on the paper. For simplicity suppose that only digits from 0 to 9 are used as symbols. Variations in handwriting are ignored. Suppose the computer is allowed 100 sheets of paper each containing 50 lines each with room for 30 digits. Then the number of states is $10^{100 \times 50 \times 30}$, i.e.~$10^{150,000}$. This is about the number of states of three Manchester machines put together. The logarithm to the base two of the number of states is usually called the `storage capacity' of the machine. Thus the Manchester machine has a storage capacity of about 165,000 and the wheel machine of our example about 1.6. If two machines are put together their capacities must be added to obtain the capacity of the resultant machine. This leads to the possibility of statements such as ``The Manchester machine contains 64 magnetic tracks each with a capacity of 2560, eight electronic tubes with a capacity of 1280. Miscellaneous storage amounts to about 300 making a total of 174,380.''\par

Given the table corresponding to a discrete state machine it is possible to predict what it will do. There is no reason why this calculation should not be carried out by means of a digital computer. Provided it could be carried out sufficiently quickly the digital computer could mimic the behaviour of any discrete state machine. The imitation game could then be played with the machine in question (as B) and the mimicking digital computer (as A) and the interrogator would be unable to distinguish them. Of course the digital computer must have an adequate storage capacity as well as working sufficiently fast. Moreover, it must be programmed afresh for each new machine which it is desired to mimic.

This special property of digital computers, that they can mimic any discrete state machine, is described by saying that they are universal machines. The existence of machines with this property has the important consequence that, considerations of speed apart, it is unnecessary to design various new machines to do various computing processes. They can all be
% --- End of Original PDF page 441 (my page 9) ---

% --- Content for Original PDF page 442 (my page 10) ---
done with one digital computer, suitably programmed for each case. It will be seen that as a consequence of this all digital computers are in a sense equivalent.

We may now consider again the point raised at the end of \S 3. It was suggested tentatively that the question, `Can machines think?' should be replaced by `Are there imaginable digital computers which would do well in the imitation game?' If we wish we can make this superficially more general and ask `Are there discrete state machines which would do well?' But in view of the universality property we see that either of these questions is equivalent to this, `Let us fix our attention on one particular digital computer C. Is it true that by modifying this computer to have an adequate storage, suitably increasing its speed of action, and providing it with an appropriate programme, C can be made to play satisfactorily the part of A in the imitation game, the part of B being taken by a man?'

% Section 6
\vspace{0.5\baselineskip} % Space before new section
\noindent\normalfont 6. \textit{Contrary Views on the Main Question.}
\vspace{0.5\baselineskip} % Small space below the section title

We may now consider the ground to have been cleared and we are ready to proceed to the debate on our question, `Can machines think?' and the variant of it quoted at the end of the last section. We cannot altogether abandon the original form of the problem, for opinions will differ as to the appropriateness of the substitution and we must at least listen to what has to be said in this connexion.

It will simplify matters for the reader if I explain first my own beliefs in the matter. Consider first the more accurate form of the question. I believe that in about fifty years' time it will be possible to programme computers, with a storage capacity of about $10^9$, to make them play the imitation game so well that an average interrogator will not have more than 70 per cent.~chance of making the right identification after five minutes of questioning. The original question, `Can machines think?' I believe to be too meaningless to deserve discussion. Nevertheless I believe that at the end of the century the use of words and general educated opinion will have altered so much that one will be able to speak of machines thinking without expecting to be contradicted. I believe further that no useful purpose is served by concealing these beliefs. The popular view that scientists proceed inexorably from well-established fact to well-established fact, never being influenced by any unproved conjecture, is quite mistaken. Provided it is made clear which are proved facts and which are conjectures, no harm can result. Conjectures are of great importance since they suggest useful lines of research.
% --- End of Original PDF page 442 (my page 10) ---

% --- Content for Original PDF page 443 (my page 11) ---
I now proceed to consider opinions opposed to my own.
\vspace{0.5\baselineskip} % Space before new sub-section

% Sub-section 1
\noindent\normalfont \textit{(1) The Theological Objection.} Thinking is a function of man's immortal soul.\footnote{\normalfont\tiny Possibly this view is heretical. St.~Thomas Aquinas (\textit{Summa Theologica}. quoted by Bertrand Russell, p.~480) states that God cannot make a man to have no soul. But this may not be a real restriction on His powers, but only a result of the fact that men's souls are immortal, and therefore indestructible.} God has given an immortal soul to every man and woman, but not to any other animal or to machines. Hence no animal or machine can think.

I am unable to accept any part of this, but will attempt to reply in theological terms. I should find the argument more convincing if animals were classed with men, for there is a greater difference, to my mind, between the typical animate and the inanimate than there is between man and the other animals. The arbitrary character of the orthodox view becomes clearer if we consider how it might appear to a member of some other religious community. How do Christians regard the Moslem view that women have no souls? But let us leave this point aside and return to the main argument. It appears to me that the argument quoted above implies a serious restriction of the omnipotence of the Almighty. It is admitted that there are certain things that He cannot do such as making one equal to two, but should we not believe that He has freedom to confer a soul on an elephant if He sees fit? We might expect that He would only exercise this power in conjunction with a mutation which provided the elephant with an appropriately improved brain to minister to the needs of this soul. An argument of exactly similar form may be made for the case of machines. It may seem different because it is more difficult to ``swallow''. But this really only means that we think it would be less likely that He would consider the circumstances suitable for conferring a soul. The circumstances in question are discussed in the rest of this paper. In attempting to construct such machines we should not be irreverently usurping His power of creating souls, any more than we are in the procreation of children: rather we are, in either case, instruments of His will providing mansions for the souls that He creates.

However, this is mere speculation. I am not very impressed with theological arguments whatever they may be used to support. Such arguments have often been found unsatisfactory in the past. In the time of Galileo it was argued that the texts, ``And the sun stood still\dots{} and hasted not to go down about a whole day'' (Joshua x.~13) and ``He laid the foundations of the earth,
% --- End of Original PDF page 443 (my page 11) ---

% --- Content for Original PDF page 444 (my page 12) ---
that it should not move at any time'' (Psalm cv.~5) were an adequate refutation of the Copernican theory. With our present knowledge such an argument appears futile. When that knowledge was not available it made a quite different impression.
\vspace{0.5\baselineskip} % Space before new sub-section

% Sub-section 2
\noindent\normalfont \textit{(2) The `Heads in the Sand' Objection.} ``The consequences of machines thinking would be too dreadful. Let us hope and believe that they cannot do so.''

This argument is seldom expressed quite so openly as in the form above. But it affects most of us who think about it at all. We like to believe that Man is in some subtle way superior to the rest of creation. It is best if he can be shown to be necessarily superior, for then there is no danger of him losing his commanding position. The popularity of the theological argument is clearly connected with this feeling. It is likely to be quite strong in intellectual people, since they value the power of thinking more highly than others, and are more inclined to base their belief in the superiority of Man on this power.

I do not think that this argument is sufficiently substantial to require refutation. Consolation would be more appropriate: perhaps this should be sought in the transmigration of souls.
\vspace{0.5\baselineskip} % Space before new sub-section

% Sub-section 3
\noindent\normalfont \textit{(3) The Mathematical Objection.} There are a number of results of mathematical logic which can be used to show that there are limitations to the powers of discrete-state machines. The best known of these results is known as \textit{Gödel}'s theorem,\footnote{\normalfont\tiny Author's names in italics refer to the Bibliography.} and shows that in any sufficiently powerful logical system statements can be formulated which can neither be proved nor disproved within the system, unless possibly the system itself is inconsistent. There are other, in some respects similar, results due to \textit{Church}, \textit{Kleene}, \textit{Rosser}, and \textit{Turing}. The latter result is the most convenient to consider, since it refers directly to machines, whereas the others can only be used in a comparatively indirect argument: for instance if \textit{Gödel}'s theorem is to be used we need in addition to have some means of describing logical systems in terms of machines, and machines in terms of logical systems. The result in question refers to a type of machine which is essentially a digital computer with an infinite capacity. It states that there are certain things that such a machine cannot do. If it is rigged up to give answers to questions as in the imitation game, there will be some questions to which it will either give a wrong answer, or fail to give an answer at all however much time is allowed for a reply. There may, of course, be many such questions, and questions which cannot be answered by one machine may be satisfactorily
% --- End of Original PDF page 444 (my page 12) ---

% --- Content for Original PDF page 445 (my page 13) ---
answered by another. We are of course supposing for the present that the questions are of the kind to which an answer `Yes' or `No' is appropriate, rather than questions such as `What do you think of Picasso?' The questions that we know the machines must fail on are of this type, ``Consider the machine specified as follows\dots{}. Will this machine ever answer `Yes' to any question?'' The dots are to be replaced by a description of some machine in a standard form, which could be something like that used in \S 5. When the machine described bears a certain comparatively simple relation to the machine which is under interrogation, it can be shown that the answer is either wrong or not forthcoming. This is the mathematical result: it is argued that it proves a disability of machines to which the human intellect is not subject.

The short answer to this argument is that although it is established that there are limitations to the powers of any particular machine, it has only been stated, without any sort of proof, that no such limitations apply to the human intellect. But I do not think this view can be dismissed quite so lightly. Whenever one of these machines is asked the appropriate critical question, and gives a definite answer, we know that this answer must be wrong, and this gives us a certain feeling of superiority. Is this feeling illusory? It is no doubt quite genuine, but I do not think too much importance should be attached to it. We too often give wrong answers to questions ourselves to be justified in being very pleased at such evidence of fallibility on the part of the machines. Further, our superiority can only be felt on such an occasion in relation to the one machine over which we have scored our petty triumph. There would be no question of triumphing simultaneously over all machines. In short, then, there might be men cleverer than any given machine, but then again there might be other machines cleverer again, and so on.

Those who hold to the mathematical argument would, I think, mostly be willing to accept the imitation game as a basis for discussion. Those who believe in the two previous objections would probably not be interested in any criteria.
\vspace{0.5\baselineskip} % Space before new sub-section

% Sub-section 4
\noindent\normalfont \textit{(4) The Argument from Consciousness.} This argument is very well expressed in Professor \textit{Jefferson}'s Lister Oration for 1949, from which I quote. ``Not until a machine can write a sonnet or compose a concerto because of thoughts and emotions felt, and not by the chance fall of symbols, could we agree that machine equals brain\textemdash that is, not only write it but know that it had written it. No mechanism could feel (and not merely
% --- End of Original PDF page 445 (my page 13) ---

% --- Content for Original PDF page 446 (my page 14) ---
artificially signal, an easy contrivance) pleasure at its successes, grief when its valves fuse, be warmed by flattery, be made miserable by its mistakes, be charmed by sex, be angry or depressed when it cannot get what it wants.''

This argument appears to be a denial of the validity of our test. According to the most extreme form of this view the only way by which one could be sure that machine thinks is to be the machine and to feel oneself thinking. One could then describe these feelings to the world, but of course no one would be justified in taking any notice. Likewise according to this view the only way to know that a man thinks is to be that particular man. It is in fact the solipsist point of view. It may be the most logical view to hold but it makes communication of ideas difficult. A is liable to believe `A thinks but B does not' whilst B believes `B thinks but A does not'. Instead of arguing continually over this point it is usual to have the polite convention that everyone thinks.

I am sure that Professor \textit{Jefferson} does not wish to adopt the extreme and solipsist point of view. Probably he would be quite willing to accept the imitation game as a test. The game (with the player B omitted) is frequently used in practice under the name of \textit{viva voce} to discover whether some one really understands something or has `learnt it parrot fashion'. Let us listen in to a part of such a \textit{viva voce}:

\vspace{0.5\baselineskip} % Space before dialogue
\dialogueturn{Interrogator}{In the first line of your sonnet which reads `Shall I compare thee to a summer's day', would not `a spring day' do as well or better?}
\dialogueturn{Witness}{It wouldn't scan.}
\dialogueturn{Interrogator}{How about `a winter's day' That would scan all right.}
\dialogueturn{Witness}{Yes, but nobody wants to be compared to a winter's day.}
\dialogueturn{Interrogator}{Would you say Mr. Pickwick reminded you of Christmas?}
\dialogueturn{Witness}{In a way.}
\dialogueturn{Interrogator}{Yet Christmas is a winter's day, and I do not think Mr. Pickwick would mind the comparison.}
\dialogueturn{Witness}{I don't think you're serious. By a winter's day one means a typical winter's day, rather than a special one like Christmas.}

And so on. What would Professor \textit{Jefferson} say if the sonnet-writing machine was able to answer like this in the \textit{viva voce}? I do not know whether he would regard the machine as `merely
% --- End of Original PDF page 446 (my page 14) ---

% --- Content for Original PDF page 447 (my page 15) ---
artificially signalling'' these answers, but if the answers were as satisfactory and sustained as in the above passage I do not think he would describe it as `an easy contrivance'. This phrase is, I think, intended to cover such devices as the inclusion in the machine of a record of someone reading a sonnet, with appropriate switching to turn it on from time to time.

In short then, I think that most of those who support the argument from consciousness could be persuaded to abandon it rather than be forced into the solipsist position. They will then probably be willing to accept our test.

I do not wish to give the impression that I think there is no mystery about consciousness. There is, for instance, something of a paradox connected with any attempt to localise it. But I do not think these mysteries necessarily need to be solved before we can answer the question with which we are concerned in this paper.
\vspace{0.5\baselineskip} % Space before new sub-section

% Sub-section 5
\noindent\normalfont \textit{(5) Arguments from Various Disabilities.} These arguments take the form, ``I grant you that you can make machines do all the things you have mentioned but you will never be able to make one to do X''. Numerous features X are suggested in this connexion. I offer a selection:

\begin{quote} % Used `quote` environment for blockquote style
Be kind, resourceful, beautiful, friendly (p.~448), have initiative, have a sense of humour, tell right from wrong, make mistakes (p.~448), fall in love, enjoy strawberries and cream (p.~448), make some one fall in love with it, learn from experience (pp.~456 f.), use words properly, be the subject of its own thought (p.~449), have as much diversity of behaviour as a man, do something really new (p.~450). (Some of these disabilities are given special consideration as indicated by the page numbers.)
\end{quote}

No support is usually offered for these statements. I believe they are mostly founded on the principle of scientific induction. A man has seen thousands of machines in his lifetime. From what he sees of them he draws a number of general conclusions. They are ugly, each is designed for a very limited purpose, when required for a minutely different purpose they are useless, the variety of behaviour of any one of them is very small, etc., etc. Naturally he concludes that these are necessary properties of machines in general. Many of these limitations are associated with the very small storage capacity of most machines. (I am assuming that the idea of storage capacity is extended in some way to cover machines other than discrete-state machines.
% --- End of Original PDF page 447 (my page 15) ---

% --- Content for Original PDF page 448 (my page 16) ---
The exact definition does not matter as no mathematical accuracy is claimed in the present discussion.) A few years ago, when very little had been heard of digital computers, it was possible to elicit much incredulity concerning them, if one mentioned their properties without describing their construction. That was presumably due to a similar application of the principle of scientific induction. These applications of the principle are of course largely unconscious. When a burnt child fears the fire and shows that he fears it by avoiding it, I should say that he was applying scientific induction. (I could of course also describe his behaviour in many other ways.) The works and customs of mankind do not seem to be very suitable material to which to apply scientific induction. A very large part of space-time must be investigated, if reliable results are to be obtained. Otherwise we may (as most English \textquotesingle Children do') decide that everybody speaks English, and that it is silly to learn French.

There are, however, special remarks to be made about many of the disabilities that have been mentioned. The inability to enjoy strawberries and cream may have struck the reader as frivolous. Possibly a machine might be made to enjoy this delicious dish, but any attempt to make one do so would be idiotic. What is important about this disability is that it contributes to some of the other disabilities, e.g.~to the difficulty of the same kind of friendliness occurring between man and machine as between white man and white man, or between black man and black man.

The claim that ``machines cannot make mistakes'' seems a curious one. One is tempted to retort, ``Are they any the worse for that?'' But let us adopt a more sympathetic attitude, and try to see what is really meant. I think this criticism can be explained in terms of the imitation game. It is claimed that the interrogator could distinguish the machine from the man simply by setting them a number of problems in arithmetic. The machine would be unmasked because of its deadly accuracy. The reply to this is simple. The machine (programmed for playing the game) would not attempt to give the right answers to the arithmetic problems. It would deliberately introduce mistakes in a manner calculated to confuse the interrogator. A mechanical fault would probably show itself through an unsuitable decision as to what sort of a mistake to make in the arithmetic. Even this interpretation of the criticism is not sufficiently sympathetic. But we cannot afford the space to go into it much further. It seems to me that this criticism depends
% --- End of Original PDF page 448 (my page 16) ---

% --- Content for Original PDF page 449 (my page 17) ---
on a confusion between two kinds of mistake. We may call them `errors of functioning' and `errors of conclusion'. Errors of functioning are due to some mechanical or electrical fault which causes the machine to behave otherwise than it was designed to do. In philosophical discussions one likes to ignore the possibility of such errors; one is therefore discussing `abstract machines'. These abstract machines are mathematical fictions rather than physical objects. By definition they are incapable of errors of functioning. In this sense we can truly say that `machines can never make mistakes'. Errors of con-clusion can only arise when some meaning is attached to the output signals from the machine. The machine might, for instance, type out mathematical equations, or sentences in English. When a false proposition is typed we say that the machine has committed an error of conclusion. There is clearly no reason at all for saying that a machine cannot make this kind of mistake. It might do nothing but type out repeatedly `0=1'. To take a less perverse example, it might have some method for drawing conclusions by scientific induction. We must expect such a method to lead occasionally to erroneous results.

The claim that a machine cannot be the subject of its own thought can of course only be answered if it can be shown that the machine has some thought with some subject matter. Nevertheless, `the subject matter of a machine's operations' does seem to mean something, at least to the people who deal with it. If, for instance, the machine was trying to find a solution of the equation $x^2 - 40x - 11 = 0$ one would be tempted to de-scribe this equation as part of the machine's subject matter at that moment. In this sort of sense a machine undoubtedly can be its own subject matter. It may be used to help in making up its own programmes, or to predict the effect of alterations in its own structure. By observing the results of its own behaviour it can modify its own programmes so as to achieve some purpose more effectively. These are possibilities of the near future, rather than Utopian dreams.

The criticism that a machine cannot have much diversity of behaviour is just a way of saying that it cannot have much storage capacity. Until fairly recently a storage capacity of even a thousand digits was very rare.

The criticisms that we are considering here are often disguised forms of the argument from consciousness. Usually if one main-tains that a machine can do one of these things, and describes the kind of method that the machine could use, one will not make
% --- End of Original PDF page 449 (my page 17) ---

% --- Content for Original PDF page 450 (my page 18) ---
much of an impression. It is thought that the method (whatever it may be, for it must be mechanical) is really rather base. Compare the parenthesis in \textit{Jefferson}'s statement quoted on p.~21.

\vspace{0.5\baselineskip} % Space before new sub-section
\noindent\normalfont \textit{(6) Lady Lovelace's Objection.} Our most detailed information of \textit{Babbage}'s Analytical Engine comes from a memoir by Lady \textit{Lovelace} (1842). In it she states, ``The Analytical Engine has no pretensions to originate anything. It can do whatever we know how to order it to perform'' (\textit{her italics}). This statement is quoted by \textit{Hartree} (p.~70) who adds: ``This does not imply that it may not be possible to construct electronic equipment which will `think for itself', or in which, in biological terms, one could set up a conditioned reflex, which would serve as a basis for `learning'.'' Whether this is possible in principle or not is a stimulating and exciting question, suggested by some of these recent developments. But it did not seem that the machines constructed or projected at the time had this property.

I am in thorough agreement with \textit{Hartree} over this. It will be noticed that he does not assert that the machines in question had not got the property, but rather that the evidence available to Lady \textit{Lovelace} did not encourage her to believe that they had it. It is quite possible that the machines in question had in a sense got this property. For suppose that some discrete-state machine has the property. The Analytical Engine was a universal digital computer, so that, if its storage capacity and speed were adequate, it could by suitable programming be made to mimic the machine in question. Probably this argument did not occur to the Countess or to \textit{Babbage}. In any case there was no obligation on them to claim all that could be claimed.

This whole question will be considered again under the heading of learning machines.

A variant of Lady \textit{Lovelace}'s objection states that a machine can `never do anything really new'. This may be parried for a moment with the saw, `There is nothing new under the sun'. Who can be certain that `original work' that he has done was not simply the growth of the seed planted in him by teaching, or the effect of following well-known general principles. A better variant of the objection says that a machine can never `take us by surprise'. This statement is a more direct challenge and can be met directly. Machines take me by surprise with great frequency. This is largely because I do not do sufficient calculation to decide what to expect them to do, or rather because, although I do a calculation, I do it in a hurried, slipshod fashion, taking risks. Perhaps I say to myself, `I suppose the voltage here ought to be the same as there: anyway let's assume it is'.
% --- End of Original PDF page 450 (my page 18) ---

% --- Content for Original PDF page 451 (my page 19) ---
Naturally I am often wrong, and the result is a surprise for me for by the time the experiment is done these assumptions have been forgotten. These admissions lay me open to lectures on the subject of my vicious ways, but do not throw any doubt on my credibility when I testify to the surprises I experience.

I do not expect this reply to silence my critic. He will probably say that such surprises are due to some creative mental act on my part, and reflect no credit on the machine. This leads us back to the argument from consciousness, and far from the idea of surprise. It is a line of argument we must consider closed, but it is perhaps worth remarking that the appreciation of some-thing as surprising requires as much of a `creative mental act' whether the surprising event originates from a man, a book, a machine or anything else.

The view that machines cannot give rise to surprises is due, I believe, to a fallacy to which philosophers and mathematicians are particularly subject. This is the assumption that as soon as a fact is presented to a mind all consequences of that fact spring into the mind simultaneously with it. It is a very use-ful assumption under many circumstances, but one too easily forgets that it is false. A natural consequence of doing so is that one then assumes that there is no virtue in the mere working out of consequences from data and general principles.
\vspace{0.5\baselineskip} % Space before new sub-section

% Sub-section 7
\noindent\normalfont \textit{(7) Argument from Continuity in the Nervous System.} The nervous system is certainly not a discrete-state machine. A small error in the information about the size of a nervous impulse impinging on a neuron, may make a large difference to the size of the outgoing impulse. It may be argued that, this being so, one cannot expect to be able to mimic the behaviour of the nervous system with a discrete-state system.

It is true that a discrete-state machine must be different from a continuous machine. But if we adhere to the conditions of the imitation game, the interrogator will not be able to take any advantage of this difference. The situation can be made clearer if we consider some other simpler continuous machine. A differential analyser will do very well. (A differential analyser is a certain kind of machine not of the discrete-state type used for some kinds of calculation.) Some of these provide their answers in a typed form, and so are suitable for taking part in the game. It would not be possible for a digital computer to predict exactly what answers the differential analyser would give to a problem, but it would be quite capable of giving the right sort of answer. For instance, if asked to give the value of $\pi$ (actually about 3.1416) it would be reasonable to choose at random between the values 3.12, 3.13, 3.14, 3.15, 3.16 with the probabilities of 0.05, 0.15, 0.55, 0.19, 0.06 (say). Under these circumstances it would be very difficult for the interrogator to distinguish the differential analyser from the digital computer.
% --- End of Original PDF page 451 (my page 19) ---

% --- Content for Original PDF page 452 (my page 20) ---
\vspace{0.5\baselineskip} % Small space before new sub-section (likely at top of page 452)
% Sub-section 8
\noindent\normalfont \textit{(8) The Argument from Informality of Behaviour.} It is not possible to produce a set of rules purporting to describe what a man should do in every conceivable set of circumstances. One might for instance have a rule that one is to stop when one sees a red traffic light, and to go if one sees a green one, but what if by some fault both appear together? One may perhaps decide that it is safest to stop. But some further difficulty may well arise from this decision later. To attempt to provide rules of conduct to cover every eventuality, even those arising from traffic lights, appears to be impossible. With all this I agree.

From this it is argued that we cannot be machines. I shall try to reproduce the argument, but I fear I shall hardly do it justice. It seems to run something like this. `If each man had a definite set of rules of conduct by which he regulated his life he would be no better than a machine. But there are no such rules, so men cannot be machines.' The undistributed middle is glaring. I do not think the argument is ever put quite like this, but I believe this is the argument used nevertheless. There may however be a certain confusion between `rules of conduct' and `laws of behaviour' to cloud the issue. By `rules of conduct' I mean precepts such as `Stop if you see red lights', on which one can act, and of which one can be conscious. By `laws of behaviour' I mean laws of nature as applied to a man's body such as `if you pinch him he will squeak'. If we substitute `laws of behaviour which regulate his life' for `laws of conduct by which he regulates his life' in the argument quoted the un-.distributed middle is no longer insuperable. For we believe that it is not only true that being regulated by laws of behaviour implies being some sort of machine (though not necessarily a discrete-state machine), but that conversely being such a machine implies being regulated by such laws. However, we cannot so easily convince ourselves of the absence of complete laws of behaviour as of complete rules of conduct. The only way we know of for finding such laws is scientific observation, and we certainly know of no circumstances under which we could say, `We have searched enough. There are no such laws.'`

We can demonstrate more forcibly that any such statement would be unjustified. For suppose we could be sure of finding

\end{document}
