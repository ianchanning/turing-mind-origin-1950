\documentclass[10pt]{article} % Starting with 10pt as a base

% --- Preamble: Packages and Global Settings ---
\usepackage[utf8]{inputenc} % Input encoding (important for direct copy-paste of text)
\usepackage[T1]{fontenc}    % Font encoding for proper glyphs and hyphenation
\usepackage{tgschola}       % For TeX Gyre Schola font
\usepackage{amsmath}        % For math environments (e.g., align)
\usepackage{amssymb}        % For additional math symbols
\usepackage{graphicx}       % For including images (if any)
\usepackage{setspace}       % For line spacing control
\usepackage{titlesec}       % For fine-grained control over section titles
\usepackage{fancyhdr}       % For custom headers and footers
\usepackage{soul}           % For letter spacing (\so)

% --- Page Geometry (from your precise measurements) ---
\usepackage[
    paperwidth=141mm,   % Width of the physical paper
    paperheight=231mm,  % Height of the physical paper
    left=22mm,          % Distance from left paper edge to text block
    right=22mm,         % Distance from right paper edge to text block
    top=22mm,           % Distance from top paper edge to text block
    bottom=34mm,        % Distance from bottom paper edge to text block
    % With these, textwidth=97mm and textheight=175mm are automatically calculated by geometry.
]{geometry}

% --- Header and Footer Configuration ---
% Define default page style using fancyhdr
\pagestyle{fancy}
\fancyhf{} % Clear all header and footer fields

% First page style (unique header/footer)
\fancypagestyle{firstpage}{
    \fancyhf{} % Clear existing settings for this page
    % Confirmed from scan: VOL. LIX. is on one line, No. 236.] is on the next, slightly indented.
    % The font size is indeed very small.
    \fancyhead[L]{\normalfont\fontsize{6.5pt}{7.5pt}\selectfont VOL. LIX.\\ \hspace*{0.5em}No. 236.]} % Top left (Vol/No info), with indent on second line
    \fancyhead[R]{\normalfont\fontsize{6.5pt}{7.5pt}\selectfont [October, 1950} % Top right (Date)
    \fancyfoot[L]{\normalfont\small 28} % Bottom left (Journal series number)
    \fancyfoot[R]{\normalfont\small 433} % Bottom right (Page number)
    \renewcommand{\headrulewidth}{0pt} % No rule under header
    \renewcommand{\footrulewidth}{0pt} % No rule above footer
}

% General page style for subsequent pages (alternating running heads and page numbers)
\fancypagestyle{mainpage}{
    \fancyhf{} % Clear existing settings
    % Left Even pages (e.g., page 434): Page number on left, Author on right
    \fancyhead[LE]{\normalfont\small \thepage \ A. M. TURING :} % Left Even
    % Right Odd pages (e.g., page 435): Article Title on left, Page number on right
    \fancyhead[RO]{\normalfont\small COMPUTING MACHINERY AND INTELLIGENCE \thepage} % Right Odd
    \renewcommand{\headrulewidth}{0pt} % No rule under header
    \renewcommand{\footrulewidth}{0pt} % No rule above footer
}
\pagestyle{mainpage} % Set this as the default style for all pages (first page overrides it)

% --- Customizing Article Title and Section Titles ---
% Custom command for the main article title (I. COMPUTING MACHINERY AND INTELLIGENCE)
% Now centered and slightly larger
\newcommand{\articletitle}[1]{%
    \vspace*{0.7cm} % Adjust vertical spacing above title
    \begin{center}% Center the title
        {\normalfont\LARGE\textbf{\uppercase{#1}}}% Slightly larger, bold, uppercase
    \end{center}%
    \par\vspace{0.2cm}%
}

% Custom command for the author line (BY A. M. TURING)
% Now centered, already in small caps
\newcommand{\articleauthor}[1]{%
    \begin{center}% Center the author
        {\normalfont\scshape #1}% Small caps
    \end{center}%
    \par\vspace{0.5cm}%
}

% Global line spacing for body text
\setstretch{1.15} % This value might need fine-tuning for exact line count per page.

% --- Document Start ---
\begin{document}
\thispagestyle{firstpage} % Apply the unique first page style to the first page only

% --- Top Journal Information (handled by fancypagestyle{firstpage}) ---

% --- Main Journal Title Block ---
\vspace*{1.2cm} % Adjust this to push "MIND" down to match the original
\begin{center}
    % Applying letter spacing with \so from 'soul' package
    % Keeping it bold and huge, as that matches the original's heavy stroke.
    {\Huge\textbf{\so{MIND}}}
\end{center}

\vspace*{0.5cm} % Space between MIND and A QUARTERLY REVIEW
\begin{center}
    {\large A QUARTERLY REVIEW} \\
    \vspace{0.1cm} % Small space between lines
    {\normalsize OF} \\
    \vspace{0.1cm} % Small space between lines
    {\large PSYCHOLOGY AND PHILOSOPHY}
\end{center}

% Decorative separator
\vspace*{0.5cm} % Space below philosophy text
\begin{center}
    \rule{3cm}{0.75pt} % Horizontal rule mimicking the decorative element
\end{center}
\vspace*{0.7cm} % Space below separator

% --- Article Title and Author ---
\articletitle{I.\textemdash COMPUTING MACHINERY AND INTELLIGENCE} % Use \textemdash for the long dash (em-dash)
\articleauthor{\textsc{By A. M. Turing}}

% --- Article Content Starts ---
% Section 1
% Section header: number normal, rest italic.
\noindent\normalfont 1. \textit{The Imitation Game.}
\vspace{0.5\baselineskip} % Small space below the section title

I \textsc{propose} to consider the question, `Can machines think?' This should begin with definitions of the meaning of the terms `machine' and `think'. The definitions might be framed so as to reflect so far as possible the normal use of the words, but this attitude is dangerous. \textsc{If} the meaning of the words `machine' and `think' are to be found by examining how they are commonly used it is difficult to escape the conclusion that the meaning and the answer to the question, `Can machines think?' is to be sought in a statistical survey such as a Gallup poll. \textsc{But} this is absurd. \textsc{Instead} of attempting such a definition \textsc{I} shall replace the question by another, which is closely related to it and is expressed in relatively unambiguous words.

\textsc{The} new form of the problem can be described in terms of a game which we call the `imitation game'. \textsc{It} is played with three people, a man (A), a woman (B), and an interrogator (C) who may be of either sex. \textsc{The} interrogator stays in a room apart from the other two. \textsc{The} object of the game for the interrogator is to determine which of the other two is the man and which is the woman. \textsc{He} knows them by labels X and Y, and at the end of the game he says either `X is A and Y is B' or `X is B and Y is A'. \textsc{The} interrogator is allowed to put questions to A and B thus:

\vspace{0.5\baselineskip} % Space before dialogue starts
\textbf{C}: Will X please tell me the length of his or her hair ?\par
\vspace{0.2\baselineskip} % Small space between dialogue turns
\textsc{Now} suppose X is actually A, then A must answer. \textsc{It} is A's
% --- End of Page 1 Content ---

% --- Page 2 Content ---
% LaTeX will naturally flow to page 2 here.
object in the game to try and cause C to make the wrong identification. \textsc{His} answer might therefore be
\vspace{0.2\baselineskip} % Small space before quoted answer
`My hair is shingled, and the longest strands are about nine inches long.'`
\vspace{0.5\baselineskip} % Space after quoted answer

\textsc{In} order that tones of voice may not help the interrogator the answers should be written, or better still, typewritten. \textsc{The} ideal arrangement is to have a teleprinter communicating between the two rooms. \textsc{Alternatively} the question and answers can be repeated by an intermediary. \textsc{The} object of the game for the third player (B) is to help the interrogator. \textsc{The} best strategy for her is probably to give truthful answers. \textsc{She} can add such things as `I am the woman, don't listen to him!'` to her answers, but it will avail nothing as the man can make similar remarks.

\textsc{We} now ask the question, `What will happen when a machine takes the part of A in this game?'` \textsc{Will} the interrogator decide wrongly as often when the game is played like this as he does when the game is played between a man and a woman? \textsc{These} questions replace our original, `Can machines think?'`

% Section 2
% Section header: number normal, rest italic.
\vspace{0.5\baselineskip} % Space before new section
\noindent\normalfont 2. \textit{Critique of the New Problem.}
\vspace{0.5\baselineskip} % Small space below the section title

\textsc{As} well as asking, `What is the answer to this new form of the question',` one may ask, `Is this new question a worthy one to investigate?'` \textsc{This} latter question \textsc{we} investigate without further ado, thereby cutting short an infinite regress.

\textsc{The} new problem has the advantage of drawing a fairly sharp line between the physical and the intellectual capacities of a man. \textsc{No} engineer or chemist claims to be able to produce a material which is indistinguishable from the human skin. \textsc{It} is possible that at some time this might be done, but even supposing this invention available \textsc{we} should feel there was little point in trying to make a `thinking machine'` more human by dressing it up in such artificial flesh. \textsc{The} form in which \textsc{we} have set the problem reflects this fact in the condition which prevents the interrogator from seeing or touching the other competitors, or hearing their voices. \textsc{Some} other advantages of the proposed criterion may be shown up by specimen questions and answers. \textsc{Thus}:

\vspace{0.5\baselineskip} % Space before dialogue
\textbf{Q}: Please write me a sonnet on the subject of the Forth Bridge.\par
\textbf{A}: Count me out on this one. \textsc{I} never could write poetry.\par
\vspace{0.2\baselineskip} % Small space between dialogue turns
\textbf{Q}: Add 34957 to 70764\par
\textbf{A}: (Pause about 30 seconds and then give as answer) 105621.\par
\vspace{0.2\baselineskip} % Small space between dialogue turns
\textbf{Q}: Do you play chess ?\par
\textbf{A}: Yes.

\end{document}
